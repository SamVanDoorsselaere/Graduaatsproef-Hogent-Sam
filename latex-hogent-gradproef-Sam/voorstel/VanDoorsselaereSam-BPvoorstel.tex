\documentclass{hogent-article}

% Invoegen bibliografiebestand
\addbibresource{voorstel.bib}

% Informatie over de opleiding, het vak en soort opdracht
\studyprogramme{Graduaat in het Programmeren}
\course{Graduaatsproef}
\assignmenttype{Onderzoeksvoorstel}

\academicyear{2025-2026}

\title{React Native mobiele app voor Energytix}

\author{Sam Van Doorsselaere}
\email{sam.vandoorsselaere@student.hogent.be}


\supervisor[Co-promotor]{K. De Ridder (IRC Engineering, \href{mailto:kdr@irc.be}{kdr@irc.be})}

\specialisation{Mobile \& Enterprise development}
\keywords{React Native, Expo, Energytix, Real-time Data, API-integratie}

\begin{document}

\begin{abstract}
Dit onderzoeksproject focust op de ontwikkeling van een mobiele applicatie voor Energytix, ter vervanging of aanvulling van de huidige webgebaseerde klok-pagina. Door gebruik te maken van React Native en Expo wordt de bestaande functionaliteit omgezet naar een native mobiele ervaring. Het doel is om gebruikers directe toegang te bieden tot real-time energieprijzen en verbruiksdata via een interface die geoptimaliseerd is voor mobiel gebruik. Het resultaat is een werkend prototype dat communiceert met de Energytix API via Axios, wat de toegankelijkheid en de algemene gebruikerservaring voor de eindgebruiker aanzienlijk verbetert.
\end{abstract}

\tableofcontents

% De hoofdtekst van het voorstel zit in een apart bestand, zodat het makkelijk
% kan opgenomen worden in de bijlagen van de graduaatsproef zelf.
\section{Context en probleemstelling}
Energytix heeft de webpagina https://klok.energytix.cloud die momenteel enkel via de browser toegankelijk is. Er is een kans om deze functionaliteit mobiel aan te bieden en zo de gebruikerservaring te verbeteren.

\section{Doelstelling}
Het doel is om de webpagina om te zetten naar React Native en te integreren in een mobiele app via Expo, waarbij gebruikers toegang krijgen tot de specifieke Energytix-data over energieprijzen en verbruik.

\section{Technologieën}
\begin{itemize}
    \item React Native
    \item Expo
    \item Axios voor API-communicatie
    \item JavaScript/TypeScript
    \item React
    \item PostGreSQL
\end{itemize}

\section{Scope}
Opleveren: een werkend prototype van de mobiele app met de functionaliteit van de klok pagina, inclusief live Energytix-data, als basis voor een mogelijke nieuwe app.

\section{Verwachte meerwaarde}
Het stagebedrijf krijgt een prototype dat gebruikers directe mobiele toegang biedt tot hun platform en real-time data, wat de toegankelijkheid en gebruikservaring verbetert.

\section{Planning en werkbelasting}
\begin{itemize}
    \item Analyse en API-verkenning: 10 uur
    \item Architectuur en ontwerp: 15 uur
    \item Opzetten project & basiscomponenten: 20 uur
    \item Omzetten webfunctionaliteit & integratie API: 35 uur
    \item Testen & debugging: 10 uur
    \item Documentatie & afronding: 10 uur
\end{itemize}



\printbibliography[heading=bibintoc]

\end{document}