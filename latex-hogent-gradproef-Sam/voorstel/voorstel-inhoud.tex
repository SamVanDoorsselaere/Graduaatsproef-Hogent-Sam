\section{Context en probleemstelling}
Energytix heeft de webpagina https://klok.energytix.cloud die momenteel enkel via de browser toegankelijk is. Er is een kans om deze functionaliteit mobiel aan te bieden en zo de gebruikerservaring te verbeteren.

\section{Doelstelling}
Het doel is om de webpagina om te zetten naar React Native en te integreren in een mobiele app via Expo, waarbij gebruikers toegang krijgen tot de specifieke Energytix-data over energieprijzen en verbruik.

\section{Technologieën}
\begin{itemize}
    \item React Native
    \item Expo
    \item Axios voor API-communicatie
    \item JavaScript/TypeScript
    \item React
    \item PostGreSQL
\end{itemize}

\section{Scope}
Opleveren: een werkend prototype van de mobiele app met de functionaliteit van de klok pagina, inclusief live Energytix-data, als basis voor een mogelijke nieuwe app.

\section{Verwachte meerwaarde}
Het stagebedrijf krijgt een prototype dat gebruikers directe mobiele toegang biedt tot hun platform en real-time data, wat de toegankelijkheid en gebruikservaring verbetert.

\section{Planning en werkbelasting}
\begin{itemize}
    \item Analyse en API-verkenning: 10 uur
    \item Architectuur en ontwerp: 15 uur
    \item Opzetten project & basiscomponenten: 20 uur
    \item Omzetten webfunctionaliteit & integratie API: 35 uur
    \item Testen & debugging: 10 uur
    \item Documentatie & afronding: 10 uur
\end{itemize}

