%%=============================================================================
%% Conclusie - Sam Van Doorsselaere
%%=============================================================================

\chapter{Conclusie}
\label{ch:conclusie}

Deze graduaatsproef realiseerde de transformatie van de Energytix-prijsinformatie naar een mobiel prototype, met als doel dynamische energieprijzen intuïtief te presenteren.

\section{Evaluatie van de doelstellingen}
Het beoogde prototype in React Native is succesvol opgeleverd en functioneel bevonden. De belangrijkste resultaten zijn:

\begin{itemize}[nosep, leftmargin=*]
    \item \textbf{Data-integratie:} De Node.js-tussenlaag verwerkt real-time prijzen uit PostgreSQL en de API efficiënt voor mobiel gebruik.
    \item \textbf{Visuele Interface:} De custom prijsklok met "morgen-cirkel" biedt proactief inzicht in verbruikskosten.
    \item \textbf{Kleurcodering:} Het algoritme voor kleurindicatoren (groen/oranje/rood) verbetert de interpretatiesnelheid aanzienlijk t.o.v. de webomgeving.
    \item \textbf{Cross-platform:} De codebase functioneert consistent op zowel Android als iOS, ondanks platform-specifieke uitdagingen.
\end{itemize}

De gekozen stack (React Native, Expo, Node.js) bleek essentieel voor een kwalitatief en tijdig resultaat.

\section{Toekomstperspectieven}
Het prototype biedt een solide basis voor verdere uitbreidingen:
\begin{description}[nosep, leftmargin=0.5cm]
    \item[1. Historische Analyse:] Grafieken voor prijsevoluties op week- of maandbasis.
    \item[2. Filters:] Mogelijkheid tot prijsfiltering op specifieke tijdsblokken.
    \item[3. Notificaties:] Push-berichten bij prijzen onder een bepaalde drempelwaarde.
    \item[4. Diensten-integratie:] Uitbouw naar een volledige klanten-app met facturatie en verbruikshistoriek.
\end{description}

\section{Slotwoord}
Mobiele technologie is cruciaal voor de moderne energiemarkt. Door prijsdata visueel te duiden, wordt de consument empowered om slimmere keuzes te maken. Deze applicatie vormt de functionele brug tussen complexe data en de dagelijkse realiteit van de gebruiker.