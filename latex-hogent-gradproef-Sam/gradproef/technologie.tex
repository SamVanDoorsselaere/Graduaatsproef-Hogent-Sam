%%=============================================================================
%% Technologie - Sam Van Doorsselaere
%%=============================================================================

\chapter{Technologie}
\label{ch:technologie}

Dit hoofdstuk licht de technologie-stack toe, gebaseerd op de expertise vanuit de opleiding en de technische behoeften van Energytix.

\section{Frontend: React Native en Expo Go}
Voor de mobiele interface is gekozen voor \textbf{React Native} met \textbf{Expo Go}. Dit framework maakt native applicaties mogelijk via JavaScript en React.



[Image of React Native architecture diagram]


\noindent\textbf{Belangrijkste voordelen:}
\begin{itemize}[nosep, leftmargin=*]
    \item \textbf{Expertise:} Gebruik van de tijdens de opleiding aangeleerde basis voor een efficiënte start.
    \item \textbf{Cross-platform:} Eén codebase voor zowel iOS als Android.
    \item \textbf{Expo Go:} Snelle ontwikkeling via directe updates naar testtoestellen zonder volledige build-cycli.
\end{itemize}

\section{Tussenlaag: Node.js en Axios}
Er is een architectuur met een \textbf{Node.js} tussenlaag (\texttt{server.js}) opgezet. De communicatie verloopt via \textbf{Axios}. De voordelen zijn:

\begin{itemize}[nosep, leftmargin=*]
    \item \textbf{Data-transformatie:} De server berekent prijs-extremen vooraf, wat de frontend ontlast.
    \item \textbf{Security:} Gevoelige API-sleutels blijven veilig op de server.
    \item \textbf{Stabiliteit:} De tussenlaag buffert wijzigingen in externe API's; enkel de server-logica behoeft aanpassing bij externe updates.
\end{itemize}

\section{Database: PostgreSQL}
Voor de opslag van tijdgebaseerde energieprijzen wordt \textbf{PostgreSQL} ingezet. Dit relationele systeem is essentieel voor het accuraat bijhouden en opvragen van de uurlijkse prijshistoriek voor de visuele klok.

\section{Ontwikkelomgeving en Uitdagingen}
De ontwikkeling vond plaats in \textbf{Visual Studio Code}. Een specifieke uitdaging was de cross-platform compatibiliteit, waarbij componenten zoals de \texttt{DateTimePicker} op iOS andere styling vereisten dan op Android. Dit leidde tot de implementatie van platform-specifieke code en bibliotheken om een uniforme gebruikerservaring te garanderen.