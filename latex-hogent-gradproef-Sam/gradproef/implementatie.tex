%%=============================================================================
%% Implementatie - Sam Van Doorsselaere
%%=============================================================================

\chapter{Implementatie}
\label{ch:implementatie}

Dit hoofdstuk behandelt de technische realisatie, met focus op de Node.js tussenlaag, de custom prijsklok in React Native en de bijbehorende kleurlogica.

\section{Backend Logica (\texttt{server.js})}
De tussenlaag fungeert als de intelligentie van de applicatie door data-verwerking van de mobiele app naar de server te verplaatsen.

\subsection{API-integratie en Data-ophaling}
Via **Axios** haalt de server elk uur prijsgegevens op van de Energytix API en slaat deze op in **PostgreSQL**. Bij een verzoek stuurt de server een geoptimaliseerde JSON-respons door naar de mobiele app.

\subsection{Berekening van Extremen}
De \texttt{server.js} berekent direct de gemiddelde uurprijs, het piek-uur en het dal-uur. Door deze logica op de server uit te voeren, blijft de frontend "lean", wat de performantie van de smartphone bevordert.

\section{Frontend Realisatie (React Native)}
De frontend is volledig custom gebouwd zonder standaard grafiek-libraries voor maximale visuele flexibiliteit.

\subsection{De Visuele Prijsklok}
De interface is opgebouwd uit drie dynamische lagen:
\begin{itemize}[nosep, leftmargin=*]
    \item \textbf{De Prijsring:} Een cirkel waarbij segmenten de uren van de dag vertegenwoordigen.
    \item \textbf{De Wijzer:} Duidt de actuele tijd aan.
    \item \textbf{De Morgen-cirkel:} Een secundaire, kleinere cirkel voor prijsvoorspellingen (zichtbaar na 14:00u).
\end{itemize}



\subsection{Kleurlogica}
Een algoritme kent kleuren toe door uren te vergelijken met het daggemiddelde: \textbf{Groen} voor prijzen onder het gemiddelde, \textbf{Rood} voor de piekmomenten, en \textbf{Oranje} voor neutrale waarden. Dit biedt de gebruiker direct handelingsperspectief.

\section{Cross-platform Uitdagingen}
De grootste uitdaging was de \texttt{DateTimePicker}. Waar Android direct voldeed, vereiste iOS specifieke styling. Dit is opgelost via platform-specifieke code:
\small\texttt{Platform.OS === 'ios' ? <IOSDatePicker /> : <AndroidDatePicker />}

\section{Styling}
Alle UI-elementen zijn gerealiseerd via \texttt{StyleSheet.create}. Dit bood de nodige precisie om de "morgen-cirkel" exact te positioneren zonder overlap, wat essentieel is voor een overzichtelijke presentatie van vandaag en morgen op één scherm.