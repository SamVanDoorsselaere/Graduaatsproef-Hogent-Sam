%%=============================================================================
%% Inleiding
%%=============================================================================

\chapter{Inleiding}%
\label{ch:inleiding}

De energiemarkt is door dynamische uurtarieven sterk veranderd. Voor Energytix is het cruciaal om deze data niet alleen aan te bieden, maar ook mobiel toegankelijk te maken. Hoewel er een webplatform bestaat, vormt een mobiele applicatie de essentiële volgende fase in hun digitale dienstverlening.

\section{Probleemstelling}%
\label{sec:probleemstelling}

De huidige "klok-pagina" (\url{https://klok.energytix.cloud}) via de browser kent beperkingen voor mobiele gebruikers: een browser is minder intuïtief en trager dan een native app, het navigeren naar een URL vormt een drempel, en de interface is niet geoptimaliseerd voor mobiele interacties. Er is dus nood aan een prototype dat de drempel verlaagt en de visuele communicatie optimaliseert voor het kleine scherm.

\section{Onderzoeksvraag}%
\label{sec:onderzoeksvraag}

Centraal staat de vraag: \textit{"Hoe kan de bestaande prijsinformatie van Energytix effectief getransformeerd worden naar een cross-platform React Native app, met behoud van visuele duidelijkheid en performantie?"}

\noindent\textbf{Deelvragen:}
\begin{itemize}[nosep, leftmargin=*]
    \item Hoe ontsluit een Node.js-tussenlaag (`server.js`) efficiënt data uit de PostgreSQL-database en API?
    \item Hoe wordt een consistente "klok-interface" gestyled voor zowel iOS als Android?
    \item Hoe kan de "morgen-cirkel" logisch geïntegreerd worden om de gebruiksvriendelijkheid te verhogen?
\end{itemize}

\section{Onderzoeksdoelstelling}%
\label{sec:onderzoeksdoelstelling}

Het resultaat is een werkend \textbf{prototype} dat stabiel verbonden is met de database, een custom gestylede klok-interface bevat en functioneel is op zowel iOS als Android (getest via Expo Go).

\section{Opzet van de graduaatsproef}%
\label{sec:opzet-graduaatsproef}

Dit verslag is als volgt opgebouwd: Hoofdstuk~\ref{ch:methodologie} bespreekt de methodologie en technologiekeuzes. Hoofdstuk~\ref{ch:analyse} behandelt de functionele behoeften en data-architectuur. Hoofdstuk~\ref{ch:technologie} gaat dieper in op de technologie-stack (Expo, Axios, PostgreSQL). In Hoofdstuk~\ref{ch:implementatie} komt de effectieve realisatie aan bod. Hoofdstuk~\ref{ch:conclusie} formuleert de conclusie en aanbevelingen.