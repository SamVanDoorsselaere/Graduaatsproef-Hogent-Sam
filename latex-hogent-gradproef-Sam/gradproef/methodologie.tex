%%=============================================================================
%% Methodologie - Sam Van Doorsselaere
%%=============================================================================

\chapter{Methodologie}%
\label{ch:methodologie}

Dit hoofdstuk licht de aanpak voor de ontwikkeling van de Energytix-applicatie toe, met de focus op fasering, technologiekeuze en kwaliteitsbewaking.

\section{Ontwikkelmethode}
Voor dit prototype is een **fasemodel** gehanteerd, gericht op een robuuste koppeling tussen bestaande data en de nieuwe mobiele interface:

\begin{description}[nosep, leftmargin=0.5cm]
    \item[Fase 1: Analyse:] Identificatie van essentiële data (uurprijzen, extremen en voorspellingen) voor mobiel gebruik.
    \item[Fase 2: Architectuur:] Opzet van de Node.js-tussenlaag (\texttt{server.js}) voor communicatie tussen PostgreSQL, de API en de app.
    \item[Fase 3: Frontend:] Ontwikkeling van de React Native interface met custom styling voor de visuele identiteit van de klok.
    \item[Fase 4: Validatie:] Uitvoerige tests op iOS en Android via Expo Go om cross-platform consistentie te waarborgen.
\end{description}

\section{Projectopvolging}
Als individueel traject lag de nadruk op zelfstandigheid. De keuze voor **React Native** en **Expo** volgde uit de expertise vanuit de opleiding. De voortgang werd getoetst aan een planning van circa 100 uur, in nauwe afstemming met de stagementor voor de API-integratie.

\section{Gebruik van Generatieve AI}
AI-tools zoals **ChatGPT** en **Gemini** werden ingezet voor:
\begin{itemize}[nosep, leftmargin=*]
    \item \textbf{Troubleshooting:} Oplossen van bugs bij Axios en asynchrone Node.js-oproepen.
    \item \textbf{Cross-platform debugging:} Zoeken naar UI-alternatieven voor consistente weergave op iOS en Android.
    \item \textbf{Documentatie:} Structureren van technische uitleg en vertaling van de abstract.
\end{itemize}
Alle AI-suggesties zijn handmatig gevalideerd en aangepast aan de Energytix-codebase.