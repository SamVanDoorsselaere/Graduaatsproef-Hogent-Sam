%%=============================================================================
%% Analyse - Sam Van Doorsselaere
%%=============================================================================

\chapter{Analyse}
\label{ch:analyse}

Voor de ontwikkeling zijn de functionele behoeften en visuele logica van Energytix in kaart gebracht om energieprijzen begrijpelijk te presenteren.

\section{Vereisten}
De applicatie moet de volgende kernfunctionaliteiten en kwalitatieve eisen bevatten:

\noindent\textbf{Functionele vereisten:}
\begin{itemize}[nosep, leftmargin=*]
    \item \textbf{Visuele Prijsklok:} Actuele uurprijs op een analoge interface voor direct begrip.
    \item \textbf{Uitersten/Gemiddelden:} Automatische berekening van het goedkoopste, duurste en gemiddelde uur.
    \item \textbf{Morgen-cirkel:} Visualisatie van prijsvoorspellingen (vanaf 14:00 uur) via een secundaire cirkel.
    \item \textbf{Kleurcodes:} Groen (voordelig), oranje (neutraal) en rood (duur) voor snelle interpretatie.
\end{itemize}

\noindent\textbf{Niet-functionele vereisten:}
\begin{itemize}[nosep, leftmargin=*]
    \item \textbf{Cross-platform:} Identieke werking op iOS en Android (specifiek datum-selectie).
    \item \textbf{Performantie:} Snelle data-ophaling via de tussenlaag zonder merkbare laadtijden.
    \item \textbf{Huisstijl:} Volledig custom styling conform de Energytix-identiteit.
\end{itemize}

\section{User Journey}
\label{sec:user_journey}

De gebruikerservaring is ontworpen voor minimale drempels:
\begin{description}[nosep, leftmargin=0.5cm]
    \item[1. Data-sync:] Bij opstarten haalt Axios data op via de \texttt{server.js} tussenlaag uit PostgreSQL of de API.
    \item[2. Hoofdscherm:] De prijsklok toont het huidige uur; de kleur van de ring geeft het prijsniveau aan.
    \item[3. Details:] Onder de klok verschijnen direct de berekende min/max/gemiddelde waarden.
    \item[4. Toekomst:] Na 14:00 uur verschijnt de 'morgen-cirkel' met een kleinere straal voor visueel onderscheid.
\end{description}



\section{Datamodel en Tussenlaag}
De backend structureert de data-stroom als volgt:
\begin{itemize}[nosep, leftmargin=*]
    \item \textbf{Prijsdata:} Ruwe uuroverzichten (timestamp en prijs).
    \item \textbf{Middleware:} De \texttt{server.js} berekent extremen vooraf om de mobiele hardware te ontlasten.
    \item \textbf{PostgreSQL:} Opslag van historische en actuele prijzen voor snelle lokale toegang.
\end{itemize}